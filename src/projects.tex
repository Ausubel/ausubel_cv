%-----------PROJECTS-----------%
\section{Proyectos}
\resumeSubHeadingListStart
    
\resumeProjectHeading
    {\textbf{\href{https://github.com/Ausubel/pokemon-restapi}{\underline{\textcolor{blue}{Proyecto Pokémon API}}}} $|$ \emph{.NET Core, Entity Framework, Microsoft SQL Server, Azure}}{}


    \resumeItemListStart
    \resumeItem{Desarrollé una api para gestionar la informacion de pokemones con una arquitectura en capas para separar responsabilidades y mejorar la modularidad.}
    \resumeItem{Implementé interfaces para desacoplar capas y usar patrones Singleton y Repository.}
    \resumeItem{Normalizacé una base de datos en SQL Server para mejorar la integridad y evitar la redundancia}
    \resumeItem{Publiqué la API en Azure con integración con una instancia de base de datos PostgreSQL}
    \resumeItemListEnd
    
    \resumeProjectHeading
    {\textbf{\href{https://github.com/Ausubel/awesone_traffick_free_project}{\underline{\textcolor{blue}{Awesone Traffick Free Project}}}} $|$ \emph{Express.js, TypeScript, Mysql, Bcrypt, JWT}}{}
    \resumeItemListStart
        \resumeItem{Desarrollé un backend para una aplicación de mercado de transferencias de jugadores de fútbol, incluyendo sistemas de inicio de sesión y registro con JWT}
        \resumeItem{El proyecto sigue una arquitectura en capas, con inyección de dependencias y patrones de diseño para mejorar la organización y mantenibilidad del código}
    \resumeItemListEnd

    \resumeProjectHeading
     {\textbf{\href{https://github.com/Ausubel/awesone_traffick_free_project}{\underline{\textcolor{blue}{Aplicación de reciclaje con gamificación – Rmap}}}} $|$ \emph{Flutter, Google Maps Platform, Firebase, Tensorflow, Keras}}{}
    \resumeItemListStart
    \resumeItem{Desarrollamos de una aplicación de reciclaje que aborda de manera integral dos Objetivos de Desarrollo Sostenible de las
Naciones Unidas: ODS 11 sobre Ciudades y Comunidades Sostenibles y ODS 13 enfocado en Acción Climática.}
       \resumeItem{Logramos ubicarnos entre los 8 primeros de un total de 42 participantes en el GDSC Perú Hack 2023.}
   \resumeItem{Desarrollé un sistema de reconocimiento de imágenes utilizando Keras y Tensorflow para la identificación y clasificación de materiales reciclables.}
    \vspace{1cm}
   \resumeItemListEnd

    \resumeProjectHeading
    {\textbf{\href{https://github.com/Ausubel/proyecto_admin}{\underline{\textcolor{blue}{Proyecto Admisión}}}} $|$ \emph{Pandas, Nunpy, Tkinter}}{}
    \resumeItemListStart
        \resumeItem{Desarrollé un sistema auxiliar para el proceso de admisión (CEPU) de la Universidad Nacional San Luis Gonzaga.}
        \resumeItem{Automatización del análisis y depuración de archivos en formato .sdf mediante Pandas, lo que mejoró la precisión y eficiencia en el procesamiento de datos.}
        \resumeItem{Implementación de una interfaz interactiva con Tkinter, proporcionando una visualización clara y accesible de los resultados.}
  
    \resumeItemListEnd
    
 \resumeProjectHeading
{\textbf{\href{https://github.com/Ausubel/cannibal_puzzle}{\underline{\textcolor{blue}{Canibal Puzzle}}}} $|$ \emph{TypeScript}}{}
\resumeItemListStart
    \resumeItem{Desarrollé un juego que consiste en resolver el famoso rompecabezas de lógica en Typescript para }
    \resumeItem{Implementé la capa principal utilizando solo un elemento HTMLCanvasElement, aprovechando todas las herramientas de este elemeneto para manejar la renderización y la lógica del juego.}
    \resumeItem{Diseñé una interfaz de usuario intuitiva y atractiva utilizando HTML, CSS y JavaScript para complementar la funcionalidad principal del juego implementada en TypeScript.}
\resumeItemListEnd
    
    

%---\resumeSubHeadingListEnd